\subsubsection{Pesquisa de mercado}
A pesquisa de mercado está detalhadamente explicada na
tabela~\ref{ape::bighatch}, no apêndice. Os seguintes robôs satisfazem os
requerimentos e restrições principais, de acordo com as tabelas~\ref{tab::bighatch} e ~\ref{tab::hvof}, e os requisitos abordados
em \ref{sec::desc_contex}: Viper s1300 (Adept), ARC Mate 100iC/12 (Fanuc),
M-10iA/12S (Fanuc), LBR iiwa 14 R820 (Kuka), KR 10 R1100 sixx WP (Kuka), MH6F-10
(Motoman), SIA10F (Motoman), MH12 (Motoman), SIA20D (Motoman). Destes, os
manipuladores LBR iiwa 14 R820 (Kuka) e Viper s1300 (Adept) deverão passar por adaptações para
operar em temperaturas até $40^o$C e umidade relativa no ar de $91\%$; e os
manipuladores KR 10 R1100 sixx WP (Kuka), MH6F-10
(Motoman) e SIA10F (Motoman) têm carga máxima de 10 Kg, que é o limite para o
processo. Dessa forma, os manipuladores comerciais prontos para o uso e que
trabalha com folga em carga são: ARC Mate 100iC/12 (Fanuc), M-10iA/12S (Fanuc),
MH12 (Motoman) e SIA20D (Motoman).

Apesar de o manipulador LBR iiwa 14 R820 (Kuka) necessitar de adaptações, seu
peso (29 Kg) representa grande vantagem perante os outros manipuladores, logo
não deve ser descartado em futuros estudos. O mesmo se pode dizer do KR 10 R1100
sixx WP (Kuka), que possui 56 Kg, mas estará operando perto de sua carga limite
(10 Kg).

Os objetos de estudo são, portanto: KR 10 R1100
sixx WP (Kuka), MH12 (Motoman), LBR iiwa 14 R820 (Kuka), ARC Mate 100iC/12
(Fanuc) e SIA20D (Motoman).


