\section{Introdução}\label{sec::introducao}
%TODO Renan: Introdução
%TODO Situar o Leitor do que foi concluido no SOTA.

Em EMMA-SOTA, é apresentada a importância da manutenção regular das turbinas em
uma usina hidrelétrica, já que, em sua operação ideal e de máxima eficiência,
sua potência tem aumento de quase 46\% após manutenção. Aumento significativo,
principalmente para países dependentes desta forma de energia, como o Brasil e
Noruega.

A eficiência de uma turbina hidrelétrica depende de inúmeras variáveis, como
volume de água, queda d'água, o tipo da turbina, o distribuidor e outras. O
projeto EMMA tem foco na manutenção do perfil hidráulico das pás dos rotores de
turbinas hidrelétricas, por este se degradar com maior rapidez, exigindo
manutenções recorrentes. 

A fim de proteger a pá contra abrasão e cavitação é realizado processo de
revestimento por asperção térmica, ou, especificamente, a metalização (HVOF).
Atualmente, este processo pode levar cerca de dois meses por turbina,
já que exige que a turbina seja desmontada, as pás serem processadas em
outro ambiente, a turbina seja remontada e recalibrada.

Apesar de o projeto visar uma solução genérica para turbinas bulbo, as
instalações são diferentes em cada usina. Desta forma, o ambiente de testes
deste projeto é a Usina Hidrelétrica de Jirau, localizada no Rio Madeira. O Rio
Madeira carrega muitos sedimentos provocando maior abrasão nas pás, se comparado
com outros usinas, além disso, a queda d'água de 2 a 20 metros intensifica o
fenômeno de cavitação. As principais características das instalações da turbina
em análise estão descritas em EMMA-SOTA, mas vale ressaltar a particularidade
dos dois acessos principais ao aro câmara, relevantes para a busca de uma
solução: acesso superior (35.7 cm de diâmetro) e acesso inferior (80 cm de
diâmetro).

O projeto EMMA busca uma solução para o processo de metalização \textit{in
situ}, isto é, revestimento das pás no ambiente da turbina, diminuindo o tempo
de manutenção e, consequentemente, de máquina parada.  A solução conceitual
desenvolvida em EMMA-SOTA é a utilização de um manipulador industrial sobre uma
base. As características do manipulador e da base variam de acordo com o
acesso: no caso da escotilha superior, a solução é um manipulador industrial de
pequeno porte e base customizada operada eletronicamente; no caso da escotilha
inferior, a solução é um manipulador industrial de porte médio e base tipo
trilho com acopladores magnéticos.

A análise das instalações da Usina Hidrelétrica de Santo Antônio, em Porto
Velho, vizinha à Jirau, mostrou que as turbinas não possuem um acesso superior.
A fim de tentar construir uma solução mais geral, o presente documento visa dar
continuidade ao projeto, detalhando o estudo de viabilidade técnica para a
solução da escotilha inferior.

