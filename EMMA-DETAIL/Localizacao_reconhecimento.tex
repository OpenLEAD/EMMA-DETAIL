\subsection{Estudo de técnicas de reconhecimento} 

As informações de distância recebidas pelos sensores descritos na seção
anterior podem ser armazenados como uma estrutura de dados chamada nuvem de
pontos, isto é, uma representação tridimensional do espaço cartesiano, na qual cada distância medida pelos
sensores a partir de sua origem representa uma coordenada x y z.
Entretanto, essa representação não é capaz de diferenciar, ou classificar, os
limites de cada objeto presente na cena, ou seja, não é possível determinar
\textit{a priori} qual conjunto de pontos pertence a cada elemento que se deseja
identificar para a realização da calibração.

A identificação de cada conjunto, ou \textit{cluster}, de pontos é importante
para que a posição e orientação de cada objeto de interesse seja determinada e,
assim a transformação do sistema de coordenadas entre cada objeto seja
calculada. Esse processo necessita, então, do estudo e implementação de
algoritmos para a análise da nuvem de pontos, identificação dos elementos
necessários, extração de suas respectivas posições e, finalmente, cálculo da
transformação entre as posições. 

Dependendo das características de cada objeto a ser identificado e da
possibilidade de implementação de uma estrutura de apoio para facilitar a sua
identificação, podem ser utilizados diferentes métodos e estratégias de
identificação e localização, que serão exploradas a seguir.

\subsubsection{Reconhecimento da Pá} 


\subsubsection{Reconhecimento do Robô}