\subsection{Estudo de técnicas de reconhecimento} 

As informações de distância recebidas pelos sensores descritos na seção
anterior podem ser armazenados como uma estrutura de dados chamada nuvem de
pontos, isto é, uma representação tridimensional do espaço cartesiano, na qual cada distância medida pelos
sensores a partir de sua origem representa uma coordenada x y z.
Entretanto, essa representação não é capaz de diferenciar, ou classificar, os
limites de cada objeto presente na cena, ou seja, não é possível determinar
\textit{a priori} qual conjunto de pontos pertence a cada elemento que se deseja
identificar para a realização da calibração.

A identificação de cada conjunto, ou \textit{cluster}, de pontos é importante
para que a posição e orientação de cada objeto de interesse seja determinada e,
assim a transformação do sistema de coordenadas entre cada objeto seja
calculada. Esse processo necessita, então, do estudo e implementação de
algoritmos para a análise da nuvem de pontos, identificação dos elementos
necessários, extração de suas respectivas posições e, finalmente, cálculo da
transformação entre as posições. 

Dependendo das características de cada objeto a ser identificado e da
possibilidade de implementação de uma estrutura de apoio para facilitar a sua
identificação, podem ser utilizados diferentes métodos e estratégias de
identificação e localização, que serão exploradas a seguir.

\subsubsection{Reconhecimento do Robô}

O Robô é uma estrutura que a idenficação pode ser facilitada pelo uso de
padrões de fácil reconhecimento (como esferas e padrões de xadrez ), pois
alterações na base do robô não ocasionam problemas para o funcionamento do sistema.

Devido a baixa iluminação ambiente dentro do circuito da tubina, a opção
mais simples é o uso de nuvem de pontos sem identificação de cor. Ou seja, o
reconhecimento se dará apenas pelo formato 


\subsubsection{Reconhecimento da Pá} 

Para a identificação e localização das pás das turbinas não é possível a
utilização de algum artifício de apoio que facilite o processamento da nuvem de
pontos, pois a instalação de qualquer um desses aparatos não pode ter a precisão
garantida nas operações de campo. Uma instalação em pontos precisos
da pá necessitaria também de calibração para cada utilização, retirando assim
o propósito do método. 

Portanto, para a localização das pás da turbina é
necessário explorar as características espaciais intrínsecas à superfície do
próprio objeto e identificá-las na nuvem de pontos do ambiente. O objetivo
principal nessa etapa do processo é identificar um conjunto mínimo de
características do objeto que represente unicamente o mesmo, com um baixo grau
de ambiguidade e sem exigir muito esforço computacional. 

A escolha do tipo de característica usar é uma decisão fundamental para a
eficiência do processo e tem sido alvo de estudos na literatura para a análise
e reconhecimentos de imagens 2D, como imagens RGB de câmeras e mais recentemente
também para imagens 3D. 

Uma boa representação de point feature deve ser capaz de capturar as mesmas
características locais da superfície na presença de:

\begin{itemize}
  \item Transformadas -  I.e, rotações e translações 3D nos dados não devem influenciar a estimação dos descriptors;
  \item Variações na densidade de amostragem - em princípio, uma de superfície
  amostrada mais ou menos densamente deve ter a mesma assinatura característica do vetor
  \item Ruído
\end{itemize}

3D haar features Pang2013

spin-image detectors - Rodgers2006

3d vs 2d Zaharia2004

superquadratic Biegelbauer2010

Range image - SIFT Bayramoglu2010
			  Local Feature Histograms  Hetzel2001
			  NARF Steder2010 Steder2009
			  Local surface patches Chen2007
			  
Intergral Image e raycasting Nuchter2005

Hough Voting - Tombari2010a
			  
			  

Compact Covariance Descriptors in 3D Point  Fehr2012

COMPARISON OF 3D INTEREST POINT DETECTORS  Weber2014




Reconhecimento de objetos

Oznabruck

Narf

Descriptors

Correspondence Grouping




\subsubsection{Reconhecimento do Robô}