\section{Estudo de viabilidade técnica detalhada}\label{sec::viatec}
O estudo de viabilidade técnica detalhada é realizado para cada acesso à turbina
(superior e inferior), como em EMMA-SOTA. O estudo passa pelas
seguintes etapas: 1) pesquisa de mercado; 2) geometria plana e/ou espacial; 3)
espaço de trabalho e cinemática do manipulador; 4) detalhamento de
bases mecânicas; 5) dinâmica do manipulador; 6) planejamento de trajetórias; 7)
sensores para calibração; e 8) técnicas de calibração.
Neste documento, serão abordadas as etapas: 1, 2, 3, 4 e 7.

A pesquisa de mercado é uma busca abrangente de soluções comerciais dentro do
escopo da solução conceitual desenvolvida no EMMA-SOTA. A pesquisa envolve
manipuladores comerciais que preencham os requisitos do processo de HVOF e
estejam de acordo com as restrições impostas pelo ambiente e o acesso. Desssa
forma, diversos fabricantes de manipuladores, Motoman, Kuka, ABB, Fanuc,
Adept e Kinova, foram avaliados e suas principais características como carga,
peso, dimensões, velocidade, temperatura e umidade de operação, são analisadas.
A pesquisa de mercado tem como objetivo retornar o objeto para os outros
estudos, ou seja, o manipulador a ser utilizado na solução.

O estudo puramente geométrico, apesar de ser diferente para cada acesso, é
genericamente uma abordagem simplificada e analítica do problema e desconsidera
alguns fatores do ambiente. O estudo geométrico tem como objetivo retornar um caso
aproximado da situação real e estimar as possíveis soluções da posição do
manipulador em relação à pá, de forma que toda ela seja revestida.

O espaço de trabalho e cinemática do manipulador é uma abordagem
detalhada e simulada. Ela considera: o meio estruturado, em um ambiente de
simulação; o manipulador com suas dimensões e limites de juntas reais;
possibilidade de colisões; real espaço de trabalho do manipulador; modelos de
bases para o manipulador; e possiveis sensores. Para a simulação é utilizada a
plataforma Openrave, uma arquitetura de planejamento para robôs autônomos,
sendo possivelmente integrada para controle em tempo real e monitoramento.
Ela provê funcionalidades para operações de cinemática direta e inversa, e
simulações físicas, e apresenta ferramentas e interfaces para planejamento de
manipuladores e um protocolo que interpreta scripts na linguagem MatLab, Octave
e Python \citep{diankov2008openrave}.

\subsection{Acesso pela escotilha inferior}
