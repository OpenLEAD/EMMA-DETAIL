\section{Estudo de viabilidade técnica detalhada}\label{sec::viatec}
O estudo de viabilidade técnica detalhada é realizado para cada acesso à turbina
(superior e inferior), como em EMMA-SOTA. O estudo passa pelas
seguintes etapas: pesquisa de mercado; geometria plana e/ou espacial; espaço de
trabalho e cinemática do manipulador; soluções de bases mecânicas; dinâmica do manipulador; e planejamento de trajetórias.

A pesquisa de mercado é uma busca abrangente de soluções comerciais dentro do
escopo da solução conceitual desenvolvida no EMMA-SOTA. A pesquisa envolve
manipuladores comerciais que preencham os requisitos do processo de HVOF e
estejam de acordo com as restrições impostas pelo ambiente e o acesso. Desssa
forma, diversos fabricantes de manipuladores, Motoman, Kuka, ABB, Fanuc,
Adept e Kinova, foram avaliados e suas principais características como carga,
peso, dimensões, velocidade, temperatura e umidade de operação, são analisadas.

\subsection{Acesso pela escotilha inferior}
