 \subsubsection{Interface de Usuários}\label{sec::interface} 

O robô EMMA prevê a construção de um software operacional que 
controlará das atividades do robô através de uma interface gráfica com o
usuário.
O objetivo desta interface é o de facilitar o processo de manutenção e aplicação
de hardcoating em turbinas através de um sistema conciso e objetivo. Para tanto
o desenvolvimento desse software foi abordado de forma ágil e centrado no
usuário, onde interações são necessárias ate que se chegue em resultados
desejados.
O processo de pesquisa e desenvolvimento desse software acontece em três fases :
descoberta, design e suporte de desenvolvimento.

\subsubsection{Descoberta}

Durante a fase de descoberta a prioridade é entender os objetivos estratégicos
do software que esta será desenvolvido. Em um primeito momento assimilar o
escopo do projeto e definindo seus requerimentos funcionais e não funcionais. 
Posteriormente aprender sobre os usuários alvo do produto, suas características
e as tarefas que estes vão executar no software através de pesquisa e
modelagem de usuários, além de uma análise detalhada de tarefas.

\paragraph{Pesquisa de usuários}
A pesquisa de usuário tem como objetivo fornecer dados suficientes para que se
crie perfis que englobem todos os possíveis atores envolvidos no
processo. No caso usuários primários, secundários e até terciários que são
agentes diretos no processo de manutenção e aplicação de hardcoating in Situ.
De forma complementar é feita uma pesquisa com um grupo representativo
destes usuários, engenheiros do laboratório e engeheiros na usina de Jirau que
utilizarão o sistema com o intuito de aprender sobre suas características, necessidades
e preferencias através de questionários e entrevistas.Esta informação e então
utilizada para construir arquétipos de usuários que atuam como referência para
tomada de decisões tanto na qrquitetura de informação quando no design de
interfaces do projeto.


\paragraph{Análise de Tarefas}
A análise de tarefas ocorre de forma simultânea a pesquisa de usuários, seu
objetivo é de compreender todas as atividades que serão executadas no software,
suas sub-tarefas e todo o caminho de trabalho feito até que se realize
uma determinada atividade. O robô EMMA contém três atividades distintas:

\begin{itemize}
  \item Calibração
  \item Planejamento de Trajetória
  \item Aplicação de Hardcoating
\end{itemize}

\paragraph{Casos de Uso}


\subsubsection{Design}

\subsubsection{Desenvolvimento}





