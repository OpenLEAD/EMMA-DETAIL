 \section{Interface de Usuários}\label{sec::interface} 

A interface gráfica do usuário permite a interação com o manipulador para a
executar as tarefas inspeção e metalização das pás in Situ. O objetivo é de facilitar a usabilidade do software dando
visibilidade aos dispositivos do sistema e monitorando o processo de metalização das pás.
O estudo de viabilidade técnica prevê seu detalhamento até seu design
conceitual, uma vez que seu desenvolvimento efetivo acontecerá posteriormente na
fase de execução do projeto EMMA.


\subsection{Pesquisa de usuários}
A pesquisa de usuário identifica todos os atores possíveis no contexto de uso do
software em questão, assim como coleta dados a respeito de seus possíveis usuários com o intuito de 
aprender sobre suas características, necessidades e preferências. Seu objtivo é
de elaborar arquétipos que sirvam como referência na tomada de decisões e na
arquitetura e design do sistema, bem como auxiliar nos requisitos funcionais e
não funcionais.


\subsection{Análise de Tarefas}
A análise de tarefas ocorre de forma simultânea a pesquisa de usuários, podendo
ser aplicado a uma variedade de técnicas para identificar e compreender a estrutura, o fluxo, 
e os atributos de tarefas executadas na metalização das pás. Seu objetivo é de
identificar as ações e processos cognitivos necessários que o usuário complete
uma tarefa ou atinja um objetivo particular. A partir desta análise de tarefas é
possível projetar e atribuir atividades de forma adequada dentro do novo sistema.
Para realizar a metalizaçào das pás in Situ o softeware do robô EMMA vai
realizar 3 tarefas distintas: calibração, planejamento de trajetória e
metalização.

\subsubsection{Calibração}
A etapa de calibração é efetuada toda vez que o robô precisa ser
posicionado ou reposicionado no ambiente do aro câmera. O objetivo é de
reconhecer no ambiente confinado todos os elementos que fazem parte da
operação. Através de um sensor laser é obtida uma nuvem de pontos que
posteriormente é analisada e fornece as posições do manipulador em relaçào a pá
e ao trilho mecânico. A partir desses dados é opossível realizar o planejar a
trajetória do robô.


\subsubsection{Planejamento de Trajetória}
Planejamento de trajetória define como o robô irá executar a aplicação de
revestimento nas pás. A partir das posições do robô em relaçào a pá são geradas
trajetórias que o efetuador irá percorrer para realizar o revestimento em toda a
superfície da pá. Cada configuração do manipulador se relaciona com uma lista de
ângulos das juntas do robô. A aplicação de metalizaçào é dividida em
etapas, para cada uma delas são geradas n configurações gerando assim uma matriz de
ângulo dessas juntas.

\subsubsection{Metalização}
A etapa de matalização conclui o processo de reparo e manutenção das pás pa
partir da aplicação de revestimento metálico. Como definido no \textit{Estudo do
conceito para metodologia e revestimento robótico de turbinas In situ}


\subsection{Casos de Uso}
Casos de uso descreve os cenários e usuários presentes nas atividades do robô.
Desta forma é possível viazualizar funcionalidades do sistema do ponto de vista
do usuário e de auxiliar na comunicação entre desenvolvedores e
clientes.

\subsection{Design Conceitual}
Design conceitual corresponde a um protótipo baseado na pesquisa de usuários,
análise de tarefas e casos de uso. É elaborado um desenho de estrutura, mais
conhecido como 'wireframe' onde se descreve as funcionalidades e posicionamento
de dispositivos. A partir são elaborados os primeiros testes de interação para
verificar seu funcionamento.

\subsection{Estrutura de Testes de Usabilidade}
Em um primeiro momento é preciso testar a estrutura do teste por meio de
observação dos usuários. Ao verificar se os testes são capazes de prover as
informações necessárias, é definido como o teste de usabilidade será aplicado.
Nesse contexto é de suma importância que todas vezes que o teste ocorra, seja
aplicado a um usuário diferente. O objetivo é que este não tenha em mente o
teste anterior a fim de garantir sua eficiência.







